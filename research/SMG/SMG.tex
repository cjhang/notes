%%%%%%%%%%%%%%%%%%%%%%%%%%%%%%%%%%%%%%%%%
% Basic notes taking example
% License:
% CC BY-NC-SA 3.0 (http://creativecommons.org/licenses/by-nc-sa/3.0/)
%
%%%%%%%%%%%%%%%%%%%%%%%%%%%%%%%%%%%%%%%%%

\documentclass{astronotes/astronotes} % Document font size and equations flushed left

\usepackage[english]{babel} % Specify a different language here - english by default

\usepackage{lipsum} % Required to insert dummy text. To be removed otherwise

%----------------------------------------------------------------------------------------
%	ARTICLE INFORMATION
%----------------------------------------------------------------------------------------

\NoteInfo{Notes} % Note information
\Archive{Daily update} % Additional notes (e.g. copyright, DOI, review/research article)

\PaperTitle{Distant, dusty star-forming galaxies} % Article title

\Authors{Jianhang Chen\textsuperscript{1}*} % Authors
\affiliation{\textsuperscript{1}\textit{European Sourthern Observatory, Garching, Germany}} % Author affiliation
\affiliation{*\textbf{Email}: cjhastro@gmail.com} % Corresponding author

\Keywords{SMG --- Starburst --- galaxy evolution} % Keywords - if you don't want any simply remove all the text between the curly brackets
\newcommand{\keywordname}{Keywords} % Defines the keywords heading name

%----------------------------------------------------------------------------------------
%	ABSTRACT
%----------------------------------------------------------------------------------------

\Abstract{This daily updated notes for the PhD research project on distant, dusty star-forming galaxies.}

%----------------------------------------------------------------------------------------
%\setcounter{tocdepth}{3}
%\makeatletter
%\def\l@subsection{\@tocline{2}{0pt}{2.5pc}{5pc}{}}
%\makeatother

\begin{document}

%\makeatletter
%\def\l@subsection{\@tocline{2}{0pt}{1pc}{5pc}{}} 
%\def\l@subsection{\@tocline{2}{0pt}{2pc}{6pc}{}}
%\makeatother
\maketitle % Output the title and abstract box

\tableofcontents % Output the contents section

\thispagestyle{empty} % Removes page numbering from the first page

%----------------------------------------------------------------------------------------
%	ARTICLE CONTENTS
%----------------------------------------------------------------------------------------

\section{Introduction} % The \section*{} command stops section numbering
%\addcontentsline{toc}{section}{Introduction} % Adds this section to the table of contents

\lipsum[1-3] % Dum text
 and some mathematics $\cos\pi=-1$ and $\alpha$ in the text\footnote{And some mathematics $\cos\pi=-1$ and $\alpha$ in the text.}.

%------------------------------------------------

\section{Gas inflow and outflow}
\subsection{Gas inflow}
It is more difficult to identify gas inflow than outflow.
Comparing outflow, inflow gas is thought to be metal poor.
From standard ${\rm \Lambda}$CDM model, pristine gas from cosmic web can feed the star formation of the host.
It can be essential for SMGs to maintain their extreme SFR.
If such kind of bulk inflow do exist in SMGs, which can also be test by their absorption characteristics.
The first detect such a strong evidence of metal-poor gas inflow by search SMGs in the surrounding of background QSOs \citep{Fu2021a}.
The system thet found is quite interesting, which a bulk of absorption lines were detected from two QSOs.
The two quasars have different distance to the SMG, but they show some consistancy in term of some metal absorption lines.

Besides, several indirect methods also suggest the existing of metal-poor gas inflow in SMGs.
The 


\subsection{Gas outflow}

\section{The fate of SMG}
\subsection{Dark matter halo}

\section{Proto-clusters}
Proto-clusters have been discovered in different wavelength. 

In submillimeter bands, previous hyper-luminous sources found by signle-dish telescope were found to host multiple members \citep{Oteo2018, Miller2018}. 
Since then, there are several projects aiming to find more similar system.

\phantomsection
\section*{Acknowledgments} % The \section*{} command stops section numbering

\addcontentsline{toc}{section}{Acknowledgments} % Adds this section to the table of contents

So long and thanks for all the fish \citet{Oteo2018} and \citet{Miller2018}.
\cite{}

%----------------------------------------------------------------------------------------
%	REFERENCE LIST
%----------------------------------------------------------------------------------------

\phantomsection
%\bibliographystyle{apalike}
\bibliographystyle{astronotes/astronotes}
\bibliography{SMG.bib}

%\addbibresource{sample.bib} % The filename of the bibliography
%\usepackage[autostyle=true]{csquotes}

%\usepackage[backend=bibtex,style=authoryear,natbib=true]{biblatex} % Use the bibtex backend with the authoryear citation style (which resembles APA)

%\addbibresource{sample.bib} % The filename of the bibliography

%\usepackage[autostyle=true]{csquotes}

%----------------------------------------------------------------------------------------

\end{document}
